\documentclass[main]{subfiles}

\begin{document}

\section{Multi source aggregated TCP Algorithm}
Let us consider a binary classification problem, and suppose we have a training dataset $Z$ and  external test data set $X$, or we randomly partition the given dataset into training ($80\%$) and external test set ($20\%$). The algorithm for aggregated TCP from multiple sources is as follows (see Figure 1).
\begin{enumerate}

\item The training data set is randomly split into K parts (disjointly) with varying sizes. For example, Let $Z = \{ z_1 , ..., z_n \} $ be the data set, then we divide the dataset into $S_1, ..., S_K$ such that $Z = \bigcup_{i=1}^K S_i$, $k_i = |S_i|$ and $n = k_1+ ...+k_K$.

\item We compute p-values using  $(X,S_i)$ for each $S_i$, say $p_i$ for each class, then we finally aggregate the k, p-values (weighted average).  

\item We repeat the step 1 and step 2 with different values of K and $k_i's$ say for $q$ times.

\item  Then we analyze the $q$ results obtained (this part is not clear yet).

\end{enumerate}


 \begin{algorithm}[H]
 \textbf{Input:}{ (training dataset:Z, test dataset:X, label set:Y, a nonconformity measure:$\mathcal{A})$}\\
 \textbf{Output:}{\textbf{ Aggregated p-values} }\\
 \textbf{Initialization\;}
 Unequal size partition: Partition training.data randomly and unequally into K parts, $S_1, ..., S_K$ ;\\ %such that $Z = \bigcup_{i=1}^K S_i$, $k_i = |S_i|$ and $n = k_1+ ...+k_K$, where n is size of the training.data \\
 \textbf{Steps\;}
 \For{each $S_i$,  $i \in \{ 1,...,K\} $ }{
 	\For{each observation $x_j \in X$}{
 		Compute p-values by using \textbf{TCP} algorithm:\\
 		 
  		$ PValues_i =  \textbf{TCP}(S_i, x_i, Y , \mathcal{A})$;\\
  	
  }
 }
 Aggregate $PValues_i$ from various sources into a set \textbf{p-values}
 \caption{Multi source aggregated TCP}
 \textbf{return \textbf{p-values}}
 \end{algorithm}
 
 \usetikzlibrary{arrows,shapes,positioning,shadows,trees}

\tikzset{
  basic/.style  = {draw, text width=2cm, drop shadow, font=\sffamily, rectangle},
  root/.style   = {basic, thin, align=center, 
                   fill=white!30, text width=5em},
  level 2/.style = {basic,  thin,align=center, fill=white!60,
                   text width=5em},
  level 3/.style = {basic, thin, align=left, fill=white!60, text width=5em}
}


\begin{center}
\begin{figure}

\begin{tikzpicture}[
  level 1/.style={sibling distance=40mm},
  edge from parent/.style={->,draw},
  >=latex]

% root of the the initial tree, level 1
\node[root] {Dataset}
% The first level, as children of the initial tree
  child {node[level 2,xshift=-100pt] (c1) {Training Data (80\%)}}
  child {node[level 3, xshift=50pt] (c2) {Test data (20\%)}};
  

% The second level, relatively positioned nodes
\begin{scope}[every node/.style={level 2}]
\node [below of = c1, xshift=-100pt, yshift=-30] (c11) {Training Data1};
\node [right of = c11, xshift=50pt] (c12) {Training Data2};
\node [right of = c12, xshift=100pt] (c13) {Training DataK};

%\node [below of = c11, yshift=-100] (m11) {Endpoint prediction probabilities using TCP};
%\node [below of = c12, yshift=-100] (m12) {Endpoint prediction probabilities using TCP};
%\node [below of = c13, yshift=-100] (m13) {Endpoint prediction probabilities using TCP};

\node [below of = c11, yshift=-100] (p11) {Estimation of p-values using TCP};
\node [below of = c12, yshift=-100] (p12) {Estimation of p-values using TCP};
\node [below of = c13, yshift=-100] (p13) {Estimation of p-values using TCP};

\node [below of = p12, xshift=30, yshift=-100] (a11) {\textbf{Aggregated p-values } };

\end{scope}

% lines from each level 1 node to every one of its "children"
%\foreach \value in {1,2,3}
 % \draw[->] (c1.195) |- (c1\value.west);

  \draw[->] (c1)->(c11);
  \draw[->] (c1)->(c12);
  \draw[->] (c1)->(c13);
\path (c12) -- node[auto=false]{\ldots} (c13);

%\draw[->] (c11)->(m11);
%  \draw[->] (c12)->(m12);
%  \draw[->] (c13)->(m13);
 % \path (m12) -- node[auto=false]{\ldots} (m13);

\draw[->] (c11)->(p11);
  \draw[->] (c12)->(p12);
  \draw[->] (c13)->(p13);
  \path (p12) -- node[auto=false]{\ldots} (p13);
  
  \draw[->] (p11)->(a11);
  \draw[->] (p12)->(a11);
  \draw[->] (p13)->(a11);
  
  \draw[->] (c2)->(p11);
  \draw[->] (c2)->(p12);
  \draw[->] (c2)->(p13);
  
\end{tikzpicture}
\caption{Multi source aggregated TCP Algorithm}
\end{figure}

\end{center}

\end{document}
